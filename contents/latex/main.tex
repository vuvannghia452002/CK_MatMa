\documentclass{beamer}
\usetheme{CambridgeUS}
\usepackage[utf8]{vietnam}
 

\usepackage{amsmath}   % For mathematical environments 
\usepackage{cleveref}  % For referencing
 


\title{Tiêu đề Bài thuyết trình}
\author{Tên Tác giả}
\date{\today}

\begin{document}

\begin{frame}
    \titlepage
\end{frame}

\begin{frame}
    \frametitle{Mục lục}
    \tableofcontents
\end{frame}

\section{Giới thiệu}

\begin{frame}
    \frametitle{Giới thiệu}
    Đây là một đoạn giới thiệu ngắn.
\end{frame}

\section{Nội dung chính}

\begin{frame}
    \frametitle{Công thức Toán học và Tham chiếu}
    Xét công thức sau:
    \begin{equation}
        E = mc^2
        \label{eq:einstein}
    \end{equation}
    Công thức \cref{eq:einstein} là công thức nổi tiếng của Einstein.
\end{frame}

\begin{frame}
    \frametitle{Định lý, Định nghĩa và Ví dụ}
    \begin{theorem}
        Đây là một ví dụ về định lý.
    \end{theorem}

    \begin{definition}
        Đây là một định nghĩa.
    \end{definition}

    \begin{proof}
        Đây là phần chứng minh.
    \end{proof}

    \begin{example}
        Đây là một ví dụ.
    \end{example}
\end{frame}

\section{Kết luận}

\begin{frame}
    \frametitle{Kết luận}
    Đây là phần kết luận của bài thuyết trình.
\end{frame}

\end{document}


% \documentclass{beamer}
% \usepackage[utf8]{vietnam}
% \usepackage{amsmath}

% \begin{document}

% \begin{frame}{Thuật toán sinh khóa}

% \begin{enumerate}
% \item Chọn hai số nguyên tố đủ lớn, $p$ và $q$.
% \item Tính toán $n = pq$ và $\phi(n) = (p - 1)(q - 1)$.
% \item Chọn một số $e$ $(1 < e < \phi(n))$ sao cho $\text{gcd}(e, \phi(n)) = 1$.

% Giá trị $e$ sẽ được sử dụng trong mã hoá.
% \item Tìm một số $d$ sao cho $ed - 1$ chia hết cho $\phi(n)$, hay nói cách khác $d = e^{-1} \mod \phi(n)$. Giá trị $d$ sẽ được sử dụng để giải mã.
% \item Công khai khóa $K^+_B = (n, e)$ và giữ bí mật khóa $K^-_B = (n, d)$.
% \end{enumerate}

% \end{frame}

% \end{document}
