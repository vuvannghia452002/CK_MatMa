\documentclass{beamer}
\usetheme{CambridgeUS}
%%%%%%%%%%%%%%%%%%%%%%%%%%%%%%%%%%%%%%%%%%%%%%%%%%%%%%%
\setbeamercolor{block title}{bg=red!80!black, fg=white}
\setbeamercolor{block body}{bg=red!10, fg=black}
%%%%%%%%%%%%%%%%%%%%%%%%%%%%%%%%%%%%%%%%%%%%%%%%%%%%%%%
\usepackage[utf8]{vietnam}

%%%%%%%%%%%%%%%%%%%%%%%%%%%%%%%%%%%%%%%%%%%%%%%%%%%%%%%
\AtBeginSection[]
{
\begin{frame}<beamer>
\frametitle{Nội dung}

\tableofcontents[
currentsection,
subsectionstyle=hide/hide,
subsubsectionstyle=hide/hide
]
\end{frame}
}
%%%%%%%%%%%%%%%%%%%%%%%%%%%%%%%%%%%%%%%%%%%%%%%%%%%%%%%
\title[{\makebox[.15\paperwidth]{MI4100 - Mật mã và độ phức tạp thuật toán}}]{Chủ đề: Mô phỏng tấn công hệ mật mã khóa công khai RSA bằng thuật toán LLL giảm lưới}
\author[Nhóm 8]{Nhóm 8}
\date[\today]{\today}
%%%%%%%%%%%%%%%%%%%%%%%%%%%%%%%%%%%%%%%%%%%%%%%%%%%%%%%
\begin{document}
%%%%%%%%%%%%%%%%%%%%%%%%%%%%%%%%%%%%%%%%%%%%%%%%%%%%%%%

% % Trang tiêu đề cần có hình ảnh pictures/HUST2.jpeg
% % Không chỉnh sửa gì
% \begin{frame}
% \begin{tikzpicture}[remember picture, overlay]
% \node[anchor=center, inner sep=0pt] at (current page.center) {\includegraphics[width=\paperwidth, height=\paperheight]{pictures/HUST2.jpeg}};
% \fill[white, opacity=0.8] (current page.south west) rectangle (current page.north east);
% \end{tikzpicture}
% \titlepage
% \end{frame}
%%%%%%%%%%%%%%%%%%%%%%%%%%%%%%%%%%%%%%%%%%%%%%%%%%%%%%%
% \begin{frame}{Danh sách thành viên}
% \begin{block}{Nhóm 8}
% \centering
% \begin{tabular} {|l|c|}
% \hline
% Họ và tên & MSSV \\
% \hline
% Nguyễn Phan Anh & 20206113 \\
% Nguyễn Việt Anh & 20206115 \\
% Nguyễn Đình Anh & 20206111 \\
% Nguyễn Thị Hoa & 20206199 \\
% Vũ Văn Nghĩa & 20206205 \\
% \hline
% \end{tabular}
% \end{block}
% \end{frame}
%%%%%%%%%%%%%%%%%%%%%%%%%%%%%%%%%%%%%%%%%%%%%%%%%%%%%%%

\end{document}

% \documentclass{beamer}
% \usepackage[utf8]{vietnam}
% \usepackage{amsmath}

% \begin{document}

% \begin{frame}{Thuật toán sinh khóa}

% \begin{enumerate}
% \item Chọn hai số nguyên tố đủ lớn, $p$ và $q$.
% \item Tính toán $n = pq$ và $\phi(n) = (p - 1)(q - 1)$.
% \item Chọn một số $e$ $(1 < e < \phi(n))$ sao cho $\text{gcd}(e, \phi(n)) = 1$.

% Giá trị $e$ sẽ được sử dụng trong mã hoá.
% \item Tìm một số $d$ sao cho $ed - 1$ chia hết cho $\phi(n)$, hay nói cách khác $d = e^{-1} \mod \phi(n)$. Giá trị $d$ sẽ được sử dụng để giải mã.
% \item Công khai khóa $K^+_B = (n, e)$ và giữ bí mật khóa $K^-_B = (n, d)$.
% \end{enumerate}

% \end{frame}

% \end{document}