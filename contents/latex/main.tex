\documentclass{beamer}
\usetheme{CambridgeUS}
%%%%%%%%%%%%%%%%%%%%%%%%%%%%%%%%%%%%%%%%%%%%%%%%%%%%%%%
\setbeamercolor{block title}{bg=red!80!black, fg=white}
\setbeamercolor{block body}{bg=red!10, fg=black}
%%%%%%%%%%%%%%%%%%%%%%%%%%%%%%%%%%%%%%%%%%%%%%%%%%%%%%%
\usepackage[utf8]{vietnam}



 

\begin{document}
 
 
 

\end{document}


% \documentclass{beamer}
% \usepackage[utf8]{vietnam}
% \usepackage{amsmath}

% \begin{document}

% \begin{frame}{Thuật toán sinh khóa}

% \begin{enumerate}
% \item Chọn hai số nguyên tố đủ lớn, $p$ và $q$.
% \item Tính toán $n = pq$ và $\phi(n) = (p - 1)(q - 1)$.
% \item Chọn một số $e$ $(1 < e < \phi(n))$ sao cho $\text{gcd}(e, \phi(n)) = 1$.

% Giá trị $e$ sẽ được sử dụng trong mã hoá.
% \item Tìm một số $d$ sao cho $ed - 1$ chia hết cho $\phi(n)$, hay nói cách khác $d = e^{-1} \mod \phi(n)$. Giá trị $d$ sẽ được sử dụng để giải mã.
% \item Công khai khóa $K^+_B = (n, e)$ và giữ bí mật khóa $K^-_B = (n, d)$.
% \end{enumerate}

% \end{frame}

% \end{document}
