

% \documentclass{beamer}
% \usepackage[utf8]{vietnam}
% \usetheme{CambridgeUS}

% % Packages for mathematical environments and references
% \usepackage{amsmath}   % For mathematical environments
% \usepackage{amsthm}    % For theorem-like environments
% \usepackage{thmtools}  % For enhanced theorem handling
% \usepackage{cleveref}  % For referencing

% % Theorem environment definitions
% \theoremstyle{plain}
% \newtheorem{theorem}{Định lý}[section]
% \newtheorem{definition}[theorem]{Định nghĩa}
% \newtheorem{proofpart}[theorem]{Chứng minh}
% \newtheorem{example}[theorem]{Ví dụ}

% \begin{document}

% \title{Tiêu đề Bài thuyết trình}
% \author{Tên Tác giả}
% \date{\today}

% \begin{frame}
%     \titlepage
% \end{frame}

% \begin{frame}
%     \frametitle{Mục lục}
%     \tableofcontents
% \end{frame}

% \section{Giới thiệu}

% \begin{frame}
%     \frametitle{Giới thiệu}
%     Đây là một đoạn giới thiệu ngắn.
% \end{frame}

% \section{Nội dung chính}

% \begin{frame}
%     \frametitle{Công thức Toán học và Tham chiếu}
%     Xét công thức sau:
%     \begin{equation}
%         E = mc^2
%         \label{eq:einstein}
%     \end{equation}
%     Công thức \cref{eq:einstein} là công thức nổi tiếng của Einstein.
% \end{frame}

% \begin{frame}
%     \frametitle{Định lý, Định nghĩa và Ví dụ}
%     \begin{theorem}
%         Đây là một ví dụ về định lý.
%     \end{theorem}

%     \begin{definition}
%         Đây là một định nghĩa.
%     \end{definition}

%     \begin{proof}
%         Đây là phần chứng minh.
%     \end{proof}

%     \begin{example}
%         Đây là một ví dụ.
%     \end{example}
% \end{frame}

% \section{Kết luận}

% \begin{frame}
%     \frametitle{Kết luận}
%     Đây là phần kết luận của bài thuyết trình.
% \end{frame}

% \end{document}
