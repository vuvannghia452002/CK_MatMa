\section{Ứng dụng tấn công RSA}

\subsection{Giới thiệu bài toán}

%%%%%%%%%%%%%%%%%%%%%%%%%%%%%%%%%%%%%%%%%%%%%%%%%%%%%%%
\begin{frame}{Ví dụ về quên mật khẩu}

\begin{itemize}
\item Ví dụ khi chúng ta yêu cầu quên mật khẩu Facebook
\item Facebook sẽ gửi thông tin khôi phục:

\begin{figure}[h]
\includegraphics[width=0.5\textwidth]{pictures/facebook.png}
\end{figure}

\item Tương tự như: \textbf{"Mật khẩu mới của bạn là ******"}
\end{itemize}

\end{frame}
%%%%%%%%%%%%%%%%%%%%%%%%%%%%%%%%%%%%%%%%%%%%%%%%%%%%%%%
\begin{frame}{Ví dụ về quên mật khẩu}

\begin{itemize}
\item Thông điệp văn bản \(m \) bao gồm \textbf{phần văn bản mẫu} và \textbf{phần bí mật}
\end{itemize}

$$
\underbrace{\text{Mật khẩu mới của bạn là}}_{\text{Phần văn bản mẫu}}
\underbrace{\text{******}}_{\text{Phần bí mật}}
$$

\begin{itemize}
\item Nếu số mũ mã hóa \(e \) trong RSA nhỏ, ta có thể sử dụng LLL để tấn công trong thời gian đa thức.
\end{itemize}

\end{frame}
%%%%%%%%%%%%%%%%%%%%%%%%%%%%%%%%%%%%%%%%%%%%%%%%%%%%%%%
\begin{frame}{Bài toán}

\begin{block}{Bài toán}
\begin{itemize}
\item Hệ mật mã RSA có:

\begin{itemize}
\item Khóa công khai (n, e) đã biết
\item Bản mã $c$ đã biết
\item Bản rõ có dạng $m = t + x$
\item Với $t$ là phần văn bản mẫu cố định đã biết
\item Với $x$ là phần bí mật thỏa mãn $|x| \leq Y$ với \(Y \in \mathbb{Z} \)
\end{itemize}
\item Để giải bài toán, chúng ta cần đi tìm $x$
\end{itemize}

\end{block}

\end{frame}
%%%%%%%%%%%%%%%%%%%%%%%%%%%%%%%%%%%%%%%%%%%%%%%%%%%%%%%
\subsection{Các bước tổng quan}

\begin{frame}{Các bước tổng quan}

\end{frame}
%%%%%%%%%%%%%%%%%%%%%%%%%%%%%%%%%%%%%%%%%%%%%%%%%%%%%%%
\subsection{Ví dụ tấn công RSA}

\begin{frame}{Ví dụ tấn công RSA}

\begin{block}{Bài toán}
\begin{itemize}
\item Hệ mật mã RSA có:

\begin{itemize}
\item Khóa công khai (n, e) = (115348777, 3)
\item Bản mã $c = 64784502$
\item Bản rõ có dạng $m = 5180 + x$
\item Với $x$ là phần bí mật thỏa mãn $|x| \leq 9$
\end{itemize}
\item Để giải bài toán, chúng ta cần đi tìm $x$
\end{itemize}

\end{block}
\end{frame}
%%%%%%%%%%%%%%%%%%%%%%%%%%%%%%%%%%%%%%%%%%%%%%%%%%%%%%%
\begin{frame}{Ví dụ tấn công RSA}

Theo hệ mật mã RSA ta biết:

\[
64784502 = (5180 + x)^3 \mod 115348777
\]

Hay:

\[
x^3 + 15540x^2 + 80497200x + 47119990 \equiv 0 \mod 115348777
\]

\end{frame}
%%%%%%%%%%%%%%%%%%%%%%%%%%%%%%%%%%%%%%%%%%%%%%%%%%%%%%%
\begin{frame}{Ví dụ tấn công RSA}

\begin{itemize}
\item Cần giải phương trình
\[
x^3 + 15540x^2 + 80497200x + 47119990 \equiv 0 \mod 115348777
\]
\item Dựa vào cơ sở lưới:
\[
\vec{b_1} = (115348777, 0, 0, 0)
\]
\[
\vec{b_2} = (0, 1038138993, 0, 0)
\]
\[
\vec{b_3} = (0, 0, 9343250937, 0)
\]
\[
\vec{b_4} = (47119990, 724474800, 1258740, 729)
\]
\end{itemize}

\end{frame}
%%%%%%%%%%%%%%%%%%%%%%%%%%%%%%%%%%%%%%%%%%%%%%%%%%%%%%%
\begin{frame}{Ví dụ tấn công RSA}

\begin{itemize}
\item Áp dụng thuật toán LLL để giảm lưới
\item Kết quả cơ sở vector mới:
\[
\vec{v_1} = (13942760, -14799933, -3334365, 2630232)
\]
\[
\vec{nghia_2} = (0, 0, 0, 0)
\]
\[
\vec{nghia_3} = (0, 0, 0, 0)
\]
\[
\vec{nghia_4} = (0, 0, 0, 0)
\]

\end{itemize}

\end{frame}
%%%%%%%%%%%%%%%%%%%%%%%%%%%%%%%%%%%%%%%%%%%%%%%%%%%%%%%
\begin{frame}{Ví dụ tấn công RSA}

\begin{itemize}
\item Ta có $\vec{v_1}$ là vector ngắn nhất của cơ sở vector mới
\[
\vec{v_1} = (13942760, -14799933, -3334365, 2630232)
\]
\item Từ đó ta có phương trình:
\[
3608x^3 - 41165x^2 - 1644437x + 13942760 = 0
\]
\item Nghiệm tìm được: $x \approx -20.314, 8.0028, 23.747$
\item Do $|x| \leq 9$, ta chọn $x = 8$ là nghiệm.
\item Kiểm tra lại: $(5180+8)^3 \mod 115348777 = 64784502$
\end{itemize}

% \begin{block}{Chú ý}
% Khi giải phương trình bậc ba, nghiệm có thể không phải là số nguyên, ta sử dụng giá trị làm tròn để tìm nghiệm.
% \end{block}

\end{frame}
%%%%%%%%%%%%%%%%%%%%%%%%%%%%%%%%%%%%%%%%%%%%%%%%%%%%%%%

%%%%%%%%%%%%%%%%%%%%%%%%%%%%%%%%%%%%%%%%%%%%%%%%%%%%%%%
\end{document}
%%%%%%%%%%%%%%%%%%%%%%%%%%%%%%%%%%%%%%%%%%%%%%%%%%%%%%%

NDA
ví dụ
thuật toán\dots