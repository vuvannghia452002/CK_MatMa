\section{Ứng dụng tấn công RSA}

%%%%%%%%%%%%%%%%%%%%%%%%%%%%%%%%%%%%%%%%%%%%%%%%%%%%%%%
\subsection{Giới thiệu bài toán}

\begin{frame}{Ví dụ về quên mật khẩu}

\begin{itemize}
\item Ví dụ khi chúng ta yêu cầu quên mật khẩu Facebook
\item Facebook sẽ gửi thông tin khôi phục:

\begin{figure}[h]
\includegraphics[width=0.5\textwidth]{pictures/facebook.png}
\end{figure}

\item Tương tự như:
\textbf{"Mật khẩu mới của bạn là ******"}
\end{itemize}

\end{frame}
%%%%%%%%%%%%%%%%%%%%%%%%%%%%%%%%%%%%%%%%%%%%%%%%%%%%%%%
\begin{frame}{Ví dụ về quên mật khẩu}

\begin{itemize}
\item Thông điệp văn bản \(m \) bao gồm \textbf{phần văn bản mẫu} và \textbf{phần bí mật}
\end{itemize}

$$
\underbrace{\text{Mật khẩu mới của bạn là}}_{\text{Phần văn bản mẫu}}
\underbrace{\text{******}}_{\text{Phần bí mật}}
$$

\begin{itemize}
\item Nếu số mũ mã hóa \(e \) trong RSA nhỏ, ta có thể sử dụng LLL để tấn công trong thời gian đa thức.
\end{itemize}

\end{frame}
%%%%%%%%%%%%%%%%%%%%%%%%%%%%%%%%%%%%%%%%%%%%%%%%%%%%%%%
\begin{frame}{Bài toán}

\begin{block}{Bài toán}

\end{block} 
 




% Giả sử hệ thống RSA có:
% <!-- Khóa công khai (n, e) đã biết-->
% <!-- Bản rõ có dạng $m = t + x$ -->

% <!-- t là phần mẫu cố định đã biết -->
% <!-- x là phần quan trọng thỏa mãn $|x| \leq Y$ với Y nguyên -->


% Ta giả sử rằng Eve biết $B, Y, n$. Vì vậy cô ấy chỉ cần tìm $x$.




\end{frame}
%%%%%%%%%%%%%%%%%%%%%%%%%%%%%%%%%%%%%%%%%%%%%%%%%%%%%%%

% \subsection{Các bước tổng quan}

% \subsection{Ví dụ tấn công RSA}
% %%%%%%%%%%%%%%%%%%%%%%%%%%%%%%%%%%%%%%%%%%%%%%%%%%%%%%%
% \end{document}

% %%%%%%%%%%%%%%%%%%%%%%%%%%%%%%%%%%%%%%%%%%%%%%%%%%%%%%%

% %! %%%%%%%%%%%%%%%%%%%%%%%%%%%%%%%%%%%%%%%%%%%%%%%%%%%%%%
% %! %%%%%%%%%%%%%%%%%%%%%%%%%%%%%%%%%%%%%%%%%%%%%%%%%%%%%%

% %%%%%%%%%%%%%%%%%%%%%%%%%%%%%%%%%%%%%%%%%%%%%%%%%%%%%%%
% \subsection{Ví dụ ứng dụng}
% % \begin{frame}{Ví dụ ứng dụng}

% Giả sử Bob có khóa công khai $(n, e) = (n, 3)$

% Khi đó bài toán là $c \equiv (B+x)^3 \quad (mod \ n)$

% Ta có thể tạo thành đa thức:
% $$
% \begin{aligned}
% f(T) & = (B+T)^3 - c = T^3 + 3BT^2 + 3B^2T + B^3 -c\\
% & \equiv T^3 + a_2T^2 + a_1T + a_0 (mod \ n).
% \end{aligned}
% $$

% Eve đang tìm $|x| \leq Y$ sao cho $f(x) \equiv 0 \quad (mod \ n)$.

% Nói cách khác, cô ấy đang tìm nghiệm nhỏ của đồng dư đa thức $f(T) \equiv 0 \quad (mod \ n)$

% % \end{frame}
% %%%%%%%%%%%%%%%%%%%%%%%%%%%%%%%%%%%%%%%%%%%%%%%%%%%%%%%
% % \begin{frame}{Ví dụ ứng dụng}

% Eve áp dụng thuật toán LLL cho lưới được tạo bởi các vector:

% $$v_1 =(n, 0, 0, 0), v_2 = (0, Yn, 0, 0) $$

% $$ v_3 = (0, 0, Y^2n, 0), v_4 = (a_0, a_1Y, a_2Y^2, Y^3)$$

% Điều này tạo nên cơ sở mới $b_1, b_2, b_3, b_4$. Nhưng ở đây, ta chỉ quan tâm đến $b_1$.

% Theo Mệnh đề 2.8 ta có:
% % Theo Mệnh đề 2.8 ta có: tham chiếu đến công thức
% % Theo Mệnh đề 2.8 ta có: tham chiếu đến công thức
% % Theo Mệnh đề 2.8 ta có: tham chiếu đến công thức
% % Theo Mệnh đề 2.8 ta có: tham chiếu đến công thức
% % Theo Mệnh đề 2.8 ta có: tham chiếu đến công thức
% % Theo Mệnh đề 2.8 ta có: tham chiếu đến công thức

% $$
% \begin{aligned}
% \|b_1\| & \leq 2^{\tfrac{3}{4}}.det(v_1, \dots, v_4)^{\tfrac{1}{4}} \qquad \quad \text{với} \quad \alpha = \frac{3}{4} \quad \text{hay} \quad \beta = 2 \\
% & = 2^{\tfrac{3}{4}}.(n^{3}Y^{6})^{\tfrac{1}{4}} = 2^{\tfrac{3}{4}}n^{\tfrac{3}{4}}Y^{\tfrac{3}{2}} \text{.}
% \end{aligned}
% $$

% % Ta có thể viết lại:
% % $$b_1 = c_1v_1 + \dots + c_4v_4 = (e_0, Ye_1, Y^2e_2, Y^3e_3)\text{.}$$
% % với $c_i$ nguyên và với $$e_0 = c_1n + c_4a_0$$ $$e_1 = c_2n + c_4a_1$$ $$e_2 = c_3n + c_4a_2$$
% % \hspace*{7cm} $e_3 = c_4$.\\
% % Ta dễ dàng thấy rằng $e_i \equiv c_4a_i\quad (mod \ n), \quad 0\leq i\leq 2$. \\

% % \end{frame}
% %%%%%%%%%%%%%%%%%%%%%%%%%%%%%%%%%%%%%%%%%%%%%%%%%%%%%%%
% % \begin{frame}{Ví dụ ứng dụng}

% Ta có thể viết lại:
% $$b_1 = c_1v_1 + \dots + c_4v_4 = (e_0, Ye_1, Y^2e_2, Y^3e_3)\text{.}$$
% với $c_i$ nguyên và với $$e_0 = c_1n + c_4a_0$$ $$e_1 = c_2n + c_4a_1$$ $$e_2 = c_3n + c_4a_2$$
% \hspace*{7cm} $e_3 = c_4$.\\
% Ta dễ dàng thấy rằng $e_i \equiv c_4a_i\quad (mod \ n), \quad 0\leq i\leq 2$. \\

% % \end{frame}
% %%%%%%%%%%%%%%%%%%%%%%%%%%%%%%%%%%%%%%%%%%%%%%%%%%%%%%%
% % \begin{frame}{Ví dụ ứng dụng}

% Ta có thể thiết lập được đa thức $$g(T) = e_3T^3 + e_2T^2 + e_1T + e_0\text{.}$$

% Khi đó, vì số nguyên $x$ thỏa mãn $f(x) \equiv 0 \quad (mod \ n) $ và vì các hệ số của $c_4f(T)$ và $g(T)$ đồng dạng với $mod(n)$ nên, $$0 \equiv c_4f(x) \equiv g(x) \quad (mod \ n)\text{.}$$

% % \end{frame}
% %%%%%%%%%%%%%%%%%%%%%%%%%%%%%%%%%%%%%%%%%%%%%%%%%%%%%%%
% % \begin{frame}{Ví dụ ứng dụng}

% Bây giờ ta giả sử rằng $$Y < 2^{-7/6}n^{1/6}$$
% Khi đó $$
% \begin{aligned}
% |g(x)| & \leq |e_0|+|e_1x|+|e_2x^2|+|e_3x^3|\\
% & \leq |e_0| + |e_1|Y + |e_2|Y^2 + |e_3|Y^3\\
% & = (1, 1, 1, 1).(|e_0|, |e_1Y|, |e_2Y^2|, |e_3Y^3|)\\
% &\leq \|(1, 1, 1, 1)\|\|(|e_0|, |e_1Y|, |e_2Y^2|, |e_3Y^3|)\| \\
% & = 2\|b_1\|\text{,}
% \end{aligned}
% $$

% % \end{frame}
% %%%%%%%%%%%%%%%%%%%%%%%%%%%%%%%%%%%%%%%%%%%%%%%%%%%%%%%
% % \begin{frame}{Ví dụ ứng dụng}

% Trong đó, bất đẳng thức cuối sử dụng bất đẳng thức Cauchy-Schwarz cho các tích vô hướng \index{tích vô hướng} ($u.v \leq \|u\|\|v\|$) . Suy ra: $$\|b_1\| \leq 2^{\tfrac{3}{4}}n^{\tfrac{3}{4}}Y^{\tfrac{3}{2}} < 2^{\tfrac{3}{4}}n^{\tfrac{3}{4}}(2^{\tfrac{-7}{6}}n^{\tfrac{1}{6}})^{\tfrac{3}{2}} = 2^{-1}n \text{.}$$
% Do đó ta có được, $$|g(x)| < n.$$
% Vì $g(x) \equiv 0 \quad (mod \ n)$, nên ta có $g(x) = 0$. Từ đó, ta có thể tìm được tối đa 3 nghiệm của $g(x)$ . Sau đó ta sẽ thử xem nó có đưa ra bản mã chính xác hay không. Do đó Eve có thể tìm thấy $x$.\\

% % \end{frame}