\documentclass{beamer}
\usetheme{CambridgeUS}
%%%%%%%%%%%%%%%%%%%%%%%%%%%%%%%%%%%%%%%%%%%%%%%%%%%%%%%
\setbeamercolor{block title}{bg=red!80!black, fg=white}
\setbeamercolor{block body}{bg=red!10, fg=black}
%%%%%%%%%%%%%%%%%%%%%%%%%%%%%%%%%%%%%%%%%%%%%%%%%%%%%%%
\usepackage[utf8]{vietnam}
\usepackage{tikz}
\usepackage{calc}
\usepackage{hyperref}
\usepackage{graphicx}
\usepackage{lipsum}
\usepackage{bookmark}
\usepackage{amsmath}
\usepackage{amssymb}

%%%%%%%%%%%%%%%%%%%%%%%%%%%%%%%%%%%%%%%%%%%%%%%%%%%%%%%
% Định nghĩa đánh công thức có số chương
\numberwithin{equation}{section}
\AtBeginSection[]{
\setcounter{section}{\thesection}
\refstepcounter{section}
}
%%%%%%%%%%%%%%%%%%%%%%%%%%%%%%%%%%%%%%%%%%%%%%%%%%%%%%%
\AtBeginSection[]
{
\begin{frame}<beamer>
\frametitle{Nội dung}

\tableofcontents[
currentsection,
subsectionstyle=hide/hide,
subsubsectionstyle=hide/hide
]
\end{frame}
}
%%%%%%%%%%%%%%%%%%%%%%%%%%%%%%%%%%%%%%%%%%%%%%%%%%%%%%%
\title[{\makebox[.15\paperwidth]{MI4100 - Mật mã và độ phức tạp thuật toán}}]{Chủ đề: Mô phỏng tấn công hệ mật mã khóa công khai RSA bằng thuật toán LLL giảm lưới}
\author[Nhóm 8]{Nhóm 8}
\date[\today]{\today}
%%%%%%%%%%%%%%%%%%%%%%%%%%%%%%%%%%%%%%%%%%%%%%%%%%%%%%%
\begin{document}
%%%%%%%%%%%%%%%%%%%%%%%%%%%%%%%%%%%%%%%%%%%%%%%%%%%%%%%
% Trang tiêu đề cần có hình ảnh pictures/HUST2.jpeg
% Không chỉnh sửa gì
\begin{frame}
\begin{tikzpicture}[remember picture, overlay]
\node[anchor=center, inner sep=0pt] at (current page.center) {\includegraphics[width=\paperwidth, height=\paperheight]{pictures/HUST2.jpeg}};
\fill[white, opacity=0.8] (current page.south west) rectangle (current page.north east);
\end{tikzpicture}
\titlepage
\end{frame}
%%%%%%%%%%%%%%%%%%%%%%%%%%%%%%%%%%%%%%%%%%%%%%%%%%%%%%%
%! %%%%%%%%%%%%%%%%%%%%%%%%%%%%%%%%%%%%%%%%%%%%%%%%%%%%%%
%! %%%%%%%%%%%%%%%%%%%%%%%%%%%%%%%%%%%%%%%%%%%%%%%%%%%%%%
%! %%%%%%%%%%%%%%%%%%%%%%%%%%%%%%%%%%%%%%%%%%%%%%%%%%%%%%
%! %%%%%%%%%%%%%%%%%%%%%%%%%%%%%%%%%%%%%%%%%%%%%%%%%%%%%%
%! %%%%%%%%%%%%%%%%%%%%%%%%%%%%%%%%%%%%%%%%%%%%%%%%%%%%%%






 
\section{Ứng dụng phương pháp lưới trong hệ mật mã RSA}
%%%%%%%%%%%%%%%%%%%%%%%%%%%%%%%%%%%%%%%%%%%%%%%%%%%%%%%
\begin{frame}
    \frametitle{Ví dụ về quên mật khẩu}
    Khi chúng ta quên mật khẩu Facebook, Facebook sẽ gửi thông tin khôi phục. Ví dụ như:

    \begin{figure}
        \includegraphics[width=0.5\textwidth]{pictures/facebook.png}
        \caption{Thông báo khôi phục mật khẩu của Facebook}
    \end{figure}

    Tương tự như:
    \textbf{Mật khẩu mới của bạn là *****}
\end{frame}
%%%%%%%%%%%%%%%%%%%%%%%%%%%%%%%%%%%%%%%%%%%%%%%%%%%%%%%

\begin{frame}
    \frametitle{Thông điệp văn bản}
    Thông điệp văn bản m bao gồm phần chữ cái mẫu và phần quan trọng.

    \begin{itemize}
        \item Nếu số mũ mã hóa \( e \) nhỏ, LLL có thể được sử dụng để tấn công RSA trong thời gian đa thức.
    \end{itemize}
\end{frame}



%! %%%%%%%%%%%%%%%%%%%%%%%%%%%%%%%%%%%%%%%%%%%%%%%%%%%%%%
%! %%%%%%%%%%%%%%%%%%%%%%%%%%%%%%%%%%%%%%%%%%%%%%%%%%%%%%
% tấn công RSA
% tấn công RSA
% tấn công RSA
% tấn công RSA
% tấn công RSA
% tấn công RSA
% tấn công RSA
% tấn công RSA
% tấn công RSA
\begin{frame}{Ứng dụng phương pháp lưới trong hệ mật mã RSA}
\begin{block}{Tấn công hệ mật mã RSA dựa trên phương pháp lưới}

Alice muốn gửi Bob một tin nhắn có dạng:

    \textbf{Mật khẩu mới của bạn là *****}

Trong bài toán này:

\begin{itemize}
\item Bản rõ có dạng $m = B + x$
\item $B$ cố định
\item $|x| \leq Y$ với Y nguyên
\end{itemize}

Ta giả sử rằng Eve biết $B, Y, n$. Vì vậy cô ấy chỉ cần tìm $x$.
\end{block}

\end{frame}
%%%%%%%%%%%%%%%%%%%%%%%%%%%%%%%%%%%%%%%%%%%%%%%%%%%%%%%
\subsection{Ví dụ ứng dụng}
\begin{frame}{Ví dụ ứng dụng}

Giả sử Bob có khóa công khai $(n, e) = (n, 3)$

Khi đó bài toán là $c \equiv (B+x)^3 \quad (mod \ n)$

Ta có thể tạo thành đa thức:
$$
\begin{aligned}
f(T) & = (B+T)^3 - c = T^3 + 3BT^2 + 3B^2T + B^3 -c\\
& \equiv T^3 + a_2T^2 + a_1T + a_0 (mod \ n).
\end{aligned}
$$

Eve đang tìm $|x| \leq Y$ sao cho $f(x) \equiv 0 \quad (mod \ n)$.

Nói cách khác, cô ấy đang tìm nghiệm nhỏ của đồng dư đa thức $f(T) \equiv 0 \quad (mod \ n)$

\end{frame}
%%%%%%%%%%%%%%%%%%%%%%%%%%%%%%%%%%%%%%%%%%%%%%%%%%%%%%%
\begin{frame}{Ví dụ ứng dụng}

Eve áp dụng thuật toán LLL cho lưới được tạo bởi các vector:

$$v_1 =(n, 0, 0, 0), v_2 = (0, Yn, 0, 0) $$

$$ v_3 = (0, 0, Y^2n, 0), v_4 = (a_0, a_1Y, a_2Y^2, Y^3)$$

Điều này tạo nên cơ sở mới $b_1, b_2, b_3, b_4$. Nhưng ở đây, ta chỉ quan tâm đến $b_1$.

Theo Mệnh đề 2.8 ta có:
% Theo Mệnh đề 2.8 ta có: tham chiếu đến công thức
% Theo Mệnh đề 2.8 ta có: tham chiếu đến công thức
% Theo Mệnh đề 2.8 ta có: tham chiếu đến công thức
% Theo Mệnh đề 2.8 ta có: tham chiếu đến công thức
% Theo Mệnh đề 2.8 ta có: tham chiếu đến công thức
% Theo Mệnh đề 2.8 ta có: tham chiếu đến công thức

$$
\begin{aligned}
\|b_1\| & \leq 2^{\tfrac{3}{4}}.det(v_1, \dots, v_4)^{\tfrac{1}{4}} \qquad \quad \text{với} \quad \alpha = \frac{3}{4} \quad \text{hay} \quad \beta = 2 \\
& = 2^{\tfrac{3}{4}}.(n^{3}Y^{6})^{\tfrac{1}{4}} = 2^{\tfrac{3}{4}}n^{\tfrac{3}{4}}Y^{\tfrac{3}{2}} \text{.}
\end{aligned}
$$

% Ta có thể viết lại:
% $$b_1 = c_1v_1 + \dots + c_4v_4 = (e_0, Ye_1, Y^2e_2, Y^3e_3)\text{.}$$
% với $c_i$ nguyên và với $$e_0 = c_1n + c_4a_0$$ $$e_1 = c_2n + c_4a_1$$ $$e_2 = c_3n + c_4a_2$$
% \hspace*{7cm} $e_3 = c_4$.\\
% Ta dễ dàng thấy rằng $e_i \equiv c_4a_i\quad (mod \ n), \quad 0\leq i\leq 2$. \\

\end{frame}
%%%%%%%%%%%%%%%%%%%%%%%%%%%%%%%%%%%%%%%%%%%%%%%%%%%%%%%
\begin{frame}{Ví dụ ứng dụng}

Ta có thể viết lại:
$$b_1 = c_1v_1 + \dots + c_4v_4 = (e_0, Ye_1, Y^2e_2, Y^3e_3)\text{.}$$
với $c_i$ nguyên và với $$e_0 = c_1n + c_4a_0$$ $$e_1 = c_2n + c_4a_1$$ $$e_2 = c_3n + c_4a_2$$
\hspace*{7cm} $e_3 = c_4$.\\
Ta dễ dàng thấy rằng $e_i \equiv c_4a_i\quad (mod \ n), \quad 0\leq i\leq 2$. \\

\end{frame}
%%%%%%%%%%%%%%%%%%%%%%%%%%%%%%%%%%%%%%%%%%%%%%%%%%%%%%%
\begin{frame}{Ví dụ ứng dụng}

Ta có thể thiết lập được đa thức $$g(T) = e_3T^3 + e_2T^2 + e_1T + e_0\text{.}$$

Khi đó, vì số nguyên $x$ thỏa mãn $f(x) \equiv 0 \quad (mod \ n) $ và vì các hệ số của $c_4f(T)$ và $g(T)$ đồng dạng với $mod(n)$ nên, $$0 \equiv c_4f(x) \equiv g(x) \quad (mod \ n)\text{.}$$

\end{frame}
%%%%%%%%%%%%%%%%%%%%%%%%%%%%%%%%%%%%%%%%%%%%%%%%%%%%%%%
\begin{frame}{Ví dụ ứng dụng}

Bây giờ ta giả sử rằng $$Y < 2^{-7/6}n^{1/6}$$
Khi đó $$
\begin{aligned}
|g(x)| & \leq |e_0|+|e_1x|+|e_2x^2|+|e_3x^3|\\
& \leq |e_0| + |e_1|Y + |e_2|Y^2 + |e_3|Y^3\\
& = (1, 1, 1, 1).(|e_0|, |e_1Y|, |e_2Y^2|, |e_3Y^3|)\\
&\leq \|(1, 1, 1, 1)\|\|(|e_0|, |e_1Y|, |e_2Y^2|, |e_3Y^3|)\| \\
& = 2\|b_1\|\text{,}
\end{aligned}
$$

\end{frame}
%%%%%%%%%%%%%%%%%%%%%%%%%%%%%%%%%%%%%%%%%%%%%%%%%%%%%%%
\begin{frame}{Ví dụ ứng dụng}

Trong đó, bất đẳng thức cuối sử dụng bất đẳng thức Cauchy-Schwarz cho các tích vô hướng \index{tích vô hướng} ($u.v \leq \|u\|\|v\|$) . Suy ra: $$\|b_1\| \leq 2^{\tfrac{3}{4}}n^{\tfrac{3}{4}}Y^{\tfrac{3}{2}} < 2^{\tfrac{3}{4}}n^{\tfrac{3}{4}}(2^{\tfrac{-7}{6}}n^{\tfrac{1}{6}})^{\tfrac{3}{2}} = 2^{-1}n \text{.}$$
Do đó ta có được, $$|g(x)| < n.$$
Vì $g(x) \equiv 0 \quad (mod \ n)$, nên ta có $g(x) = 0$. Từ đó, ta có thể tìm được tối đa 3 nghiệm của $g(x)$ . Sau đó ta sẽ thử xem nó có đưa ra bản mã chính xác hay không. Do đó Eve có thể tìm thấy $x$.\\

\end{frame}
%%%%%%%%%%%%%%%%%%%%%%%%%%%%%%%%%%%%%%%%%%%%%%%%%%%%%%%


%! %%%%%%%%%%%%%%%%%%%%%%%%%%%%%%%%%%%%%%%%%%%%%%%%%%%%%%
%! %%%%%%%%%%%%%%%%%%%%%%%%%%%%%%%%%%%%%%%%%%%%%%%%%%%%%%
%! %%%%%%%%%%%%%%%%%%%%%%%%%%%%%%%%%%%%%%%%%%%%%%%%%%%%%%
%! %%%%%%%%%%%%%%%%%%%%%%%%%%%%%%%%%%%%%%%%%%%%%%%%%%%%%%
%! %%%%%%%%%%%%%%%%%%%%%%%%%%%%%%%%%%%%%%%%%%%%%%%%%%%%%%
\section*{}
\begin{frame}{}
\centering
\Huge{Thanks for listening!}
\end{frame}
%%%%%%%%%%%%%%%%%%%%%%%%%%%%%%%%%%%%%%%%%%%%%%%%%%%%%%%
\end{document}
%%%%%%%%%%%%%%%%%%%%%%%%%%%%%%%%%%%%%%%%%%%%%%%%%%%%%%%

% Lý thuyết
% Lịch sử
% Giới thiệu
% Mục đích công dụng

% Hình ảnh liên quan

% Thuật toán
% Các bước
% Sơ đồ thuật toán

% Code mã nguồn mã giả....

% Ví dụ minh họa
% Chạy tay
% Chương trình lập trình