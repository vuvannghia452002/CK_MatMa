\documentclass{beamer}
\usetheme{CambridgeUS}
%%%%%%%%%%%%%%%%%%%%%%%%%%%%%%%%%%%%%%%%%%%%%%%%%%%%%%%
\setbeamercolor{block title}{bg=red!80!black, fg=white}
\setbeamercolor{block body}{bg=red!10, fg=black}
%%%%%%%%%%%%%%%%%%%%%%%%%%%%%%%%%%%%%%%%%%%%%%%%%%%%%%%
\usepackage[utf8]{vietnam}
\usepackage{tikz}
\usepackage{calc}
\usepackage{hyperref}
\usepackage{graphicx}
\usepackage{lipsum}
\usepackage{bookmark}
\usepackage{amsmath}
\usepackage{amssymb}

%%%%%%%%%%%%%%%%%%%%%%%%%%%%%%%%%%%%%%%%%%%%%%%%%%%%%%%
% Định nghĩa đánh công thức có số chương
\numberwithin{equation}{section}
\AtBeginSection[]{
\setcounter{section}{\thesection}
\refstepcounter{section}
}
%%%%%%%%%%%%%%%%%%%%%%%%%%%%%%%%%%%%%%%%%%%%%%%%%%%%%%%
\AtBeginSection[]
{
\begin{frame}<beamer>
\frametitle{Nội dung}

\tableofcontents[
currentsection,
subsectionstyle=hide/hide,
subsubsectionstyle=hide/hide
]
\end{frame}
}
%%%%%%%%%%%%%%%%%%%%%%%%%%%%%%%%%%%%%%%%%%%%%%%%%%%%%%%
\title[{\makebox[.15\paperwidth]{MI4100 - Mật mã và độ phức tạp thuật toán}}]{Chủ đề: Mô phỏng tấn công hệ mật mã khóa công khai RSA bằng thuật toán LLL giảm lưới}
\author[Nhóm 8]{Nhóm 8}
\date[\today]{\today}
%%%%%%%%%%%%%%%%%%%%%%%%%%%%%%%%%%%%%%%%%%%%%%%%%%%%%%%
\begin{document}
%%%%%%%%%%%%%%%%%%%%%%%%%%%%%%%%%%%%%%%%%%%%%%%%%%%%%%%
% Trang tiêu đề cần có hình ảnh pictures/HUST2.jpeg
% Không chỉnh sửa gì
\begin{frame}
\begin{tikzpicture}[remember picture, overlay]
\node[anchor=center, inner sep=0pt] at (current page.center) {\includegraphics[width=\paperwidth, height=\paperheight]{pictures/HUST2.jpeg}};
\fill[white, opacity=0.8] (current page.south west) rectangle (current page.north east);
\end{tikzpicture}
\titlepage
\end{frame}
%%%%%%%%%%%%%%%%%%%%%%%%%%%%%%%%%%%%%%%%%%%%%%%%%%%%%%%
%! %%%%%%%%%%%%%%%%%%%%%%%%%%%%%%%%%%%%%%%%%%%%%%%%%%%%%%
%! %%%%%%%%%%%%%%%%%%%%%%%%%%%%%%%%%%%%%%%%%%%%%%%%%%%%%%
%! %%%%%%%%%%%%%%%%%%%%%%%%%%%%%%%%%%%%%%%%%%%%%%%%%%%%%%
%! %%%%%%%%%%%%%%%%%%%%%%%%%%%%%%%%%%%%%%%%%%%%%%%%%%%%%%
%! %%%%%%%%%%%%%%%%%%%%%%%%%%%%%%%%%%%%%%%%%%%%%%%%%%%%%%

%%%%%%%%%%%%%%%%%%%%%%%%%%%%%%%%%%%%%%%%%%%%%%%%%%%%%%%

%%%%%%%%%%%%%%%%%%%%%%%%%%%%%%%%%%%%%%%%%%%%%%%%%%%%%%%

% <!-- Điều kiện 1 -->

% <!-- Điều kiện 2 -->

% <!-- Mã giả -->

% <!-- Ví dụ -->

% <!--! Thuật toán LLL -->

% <!-- quy trình Gram-Schmidt: -->
% <!--Nếu $x_1, x_2, \dots, x_n$ là một cơ sở của lưới $L$ thì sau khi trực giao hóa ta thu được các vector $x_1^*, x_2^*, \dots, x_n^*$ có thể không nằm trong lưới $L$. -->
% <!-- Vì num là phân số.... -->
% <!-- Xin chào! Bạn có thể vui lòng giải thích mục đích của dòng: bk = bk - [uk, j]bj là gì không? -->
% <!-- quy trình Gram Schmidt làm cho cơ sở trực giao -->
% <!-- Tuy nhiên, trong LLL chúng ta đang làm việc trong một mạng nên không thể đảm bảo tính trực giao. -->
% <!-- Để làm được điều đó, chúng ta cần u_{k, j} là một số nguyên. -->
% <!-- Điều này tạo ra một cơ sở "đủ trực giao" trong khi vẫn còn trong mạng -->

% <!-- 2 chiều -->
% <!-- n chiều -->
% <!-- Thuật toán LLL giảm lưới -->

%! %%%%%%%%%%%%%%%%%%%%%%%%%%%%%%%%%%%%%%%%%%%%%%%%%%%%%%
%! %%%%%%%%%%%%%%%%%%%%%%%%%%%%%%%%%%%%%%%%%%%%%%%%%%%%%%
%! %%%%%%%%%%%%%%%%%%%%%%%%%%%%%%%%%%%%%%%%%%%%%%%%%%%%%%
%! %%%%%%%%%%%%%%%%%%%%%%%%%%%%%%%%%%%%%%%%%%%%%%%%%%%%%%
%! %%%%%%%%%%%%%%%%%%%%%%%%%%%%%%%%%%%%%%%%%%%%%%%%%%%%%%
\section*{}
\begin{frame}{}
\centering
\Huge{Thanks for listening!}
\end{frame}
%%%%%%%%%%%%%%%%%%%%%%%%%%%%%%%%%%%%%%%%%%%%%%%%%%%%%%%
\end{document}
%%%%%%%%%%%%%%%%%%%%%%%%%%%%%%%%%%%%%%%%%%%%%%%%%%%%%%%

% Lý thuyết
% Lịch sử
% Giới thiệu
% Mục đích công dụng

% Hình ảnh liên quan

% Thuật toán
% Các bước
% Sơ đồ thuật toán

% Code mã nguồn mã giả....

% Ví dụ minh họa
% Chạy tay
% Chương trình lập trình