\documentclass{beamer}
\usetheme{CambridgeUS}
%%%%%%%%%%%%%%%%%%%%%%%%%%%%%%%%%%%%%%%%%%%%%%%%%%%%%%%
\setbeamercolor{block title}{bg=red!80!black, fg=white}
\setbeamercolor{block body}{bg=red!10, fg=black}
%%%%%%%%%%%%%%%%%%%%%%%%%%%%%%%%%%%%%%%%%%%%%%%%%%%%%%%
\usepackage[utf8]{vietnam}
\usepackage{tikz}
\usepackage{hyperref}
\usepackage{graphicx}
\usepackage{lipsum}
\usepackage{bookmark}
\usepackage{amsmath}
\usepackage{amssymb}

%%%%%%%%%%%%%%%%%%%%%%%%%%%%%%%%%%%%%%%%%%%%%%%%%%%%%%%
% Định nghĩa đánh công thức có số chương
\numberwithin{equation}{section}
\AtBeginSection[]{
\setcounter{section}{\thesection}
\refstepcounter{section}
}
%%%%%%%%%%%%%%%%%%%%%%%%%%%%%%%%%%%%%%%%%%%%%%%%%%%%%%%
\AtBeginSection[]
{
\begin{frame}<beamer>
\frametitle{Nội dung}

\tableofcontents[
currentsection,
subsectionstyle=hide/hide,
subsubsectionstyle=hide/hide
]
\end{frame}
}
%%%%%%%%%%%%%%%%%%%%%%%%%%%%%%%%%%%%%%%%%%%%%%%%%%%%%%%
\title[{\makebox[.15\paperwidth]{MI4100 - Mật mã và độ phức tạp thuật toán}}]{Chủ đề: Mô phỏng tấn công hệ mật mã khóa công khai RSA bằng thuật toán LLL giảm lưới}
\author[Nhóm 8]{Nhóm 8}
\date[\today]{\today}
%%%%%%%%%%%%%%%%%%%%%%%%%%%%%%%%%%%%%%%%%%%%%%%%%%%%%%%
\begin{document}
%%%%%%%%%%%%%%%%%%%%%%%%%%%%%%%%%%%%%%%%%%%%%%%%%%%%%%%
% % Trang tiêu đề cần có hình ảnh pictures/HUST2.jpeg
% % Không chỉnh sửa gì
% \begin{frame}
% \begin{tikzpicture}[remember picture, overlay]
% \node[anchor=center, inner sep=0pt] at (current page.center) {\includegraphics[width=\paperwidth, height=\paperheight]{pictures/HUST2.jpeg}};
% \fill[white, opacity=0.8] (current page.south west) rectangle (current page.north east);
% \end{tikzpicture}
% \titlepage
% \end{frame}
%%%%%%%%%%%%%%%%%%%%%%%%%%%%%%%%%%%%%%%%%%%%%%%%%%%%%%%
%! %%%%%%%%%%%%%%%%%%%%%%%%%%%%%%%%%%%%%%%%%%%%%%%%%%%%%%
%! %%%%%%%%%%%%%%%%%%%%%%%%%%%%%%%%%%%%%%%%%%%%%%%%%%%%%%
%! %%%%%%%%%%%%%%%%%%%%%%%%%%%%%%%%%%%%%%%%%%%%%%%%%%%%%%
%! %%%%%%%%%%%%%%%%%%%%%%%%%%%%%%%%%%%%%%%%%%%%%%%%%%%%%%
%! %%%%%%%%%%%%%%%%%%%%%%%%%%%%%%%%%%%%%%%%%%%%%%%%%%%%%%
\section{Phương pháp lưới và thuật toán LLL}
\subsection{Phương pháp lưới}
\begin{frame}{Định nghĩa lưới}

\begin{block}{Định nghĩa lưới}

Cho \(n \geq 1 \), \(\{x_1, x_2, \ldots, x_n\}\) là một cơ sở của \(\mathbb{R}^n\).
Lưới \(n \) chiều với cơ sở \(\{x_1, x_2, \ldots, x_n\}\)
là tập hợp \(L \) tất cả các tổ hợp tuyến tính của các vector cơ sở đó với hệ số nguyên:

$$
L = \{a_1 x_1 + a_2 x_2 + \ldots + a_n x_n \mid a_i \in \mathbb{Z} \}
$$

Các vector \(\{x_1, x_2, \ldots, x_n\}\) được gọi là cơ sở của lưới.

\end{block}

\end{frame}

% ! Ví dụ lưới 2 chiều
% https://www.youtube.com/watch?v=UU2EaVXkKLY&list=PL6hzlGxGIS1A-o2pQVXK-Z2qOBOvZ1XbZ
% @ Ví dụ ảnh lưới 2 chiều b1, b2 =>3 4 5 6 7

% @ ??????????????? Nguyên, +-1, det,

% !Ứng dụng, mở đầu, ....
%

% \section{Phương pháp lưới}
% \begin{frame}{Phương pháp lưới}
% \begin{itemize}
% \item Phương pháp lưới là một lĩnh vực trong toán học có liên quan đến việc nghiên cứu các cấu trúc đại số và hình học của các mạng lưới được phát triển từ những năm 1940.
% \item Phương pháp lưới được sử dụng trong nhiều lĩnh vực:
% \begin{itemize}
% \item Lĩnh vực xấp xỉ số đại số
% \item Lĩnh vực mật mã học
% \item Lĩnh vực khoa học máy tính
% \item Lĩnh vực kỹ thuật thông tin
% \end{itemize}
% \end{itemize}
% \end{frame}

%! %%%%%%%%%%%%%%%%%%%%%%%%%%%%%%%%%%%%%%%%%%%%%%%%%%%%%%
%! %%%%%%%%%%%%%%%%%%%%%%%%%%%%%%%%%%%%%%%%%%%%%%%%%%%%%%
%! %%%%%%%%%%%%%%%%%%%%%%%%%%%%%%%%%%%%%%%%%%%%%%%%%%%%%%
%! %%%%%%%%%%%%%%%%%%%%%%%%%%%%%%%%%%%%%%%%%%%%%%%%%%%%%%
%! %%%%%%%%%%%%%%%%%%%%%%%%%%%%%%%%%%%%%%%%%%%%%%%%%%%%%%
\section*{}
\begin{frame}{}
\centering
\Huge{Thanks for listening!}
\end{frame}
%%%%%%%%%%%%%%%%%%%%%%%%%%%%%%%%%%%%%%%%%%%%%%%%%%%%%%%
\end{document}
%%%%%%%%%%%%%%%%%%%%%%%%%%%%%%%%%%%%%%%%%%%%%%%%%%%%%%%